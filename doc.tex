% !TEX TS-program = pdflatex
% !TEX encoding = UTF-8 Unicode

% This is a simple template for a LaTeX document using the "article" class.
% See "book", "report", "letter" for other types of document.

\documentclass[11pt]{article} % use larger type; default would be 10pt

\usepackage[utf8]{inputenc} % set input encoding (not needed with XeLaTeX)

%%% Examples of Article customizations
% These packages are optional, depending whether you want the features they provide.
% See the LaTeX Companion or other references for full information.

%%% PAGE DIMENSIONS
\usepackage{geometry} % to change the page dimensions
\geometry{a4paper} % or letterpaper (US) or a5paper or....
% \geometry{margin=2in} % for example, change the margins to 2 inches all round
% \geometry{landscape} % set up the page for landscape
%   read geometry.pdf for detailed page layout information

\usepackage{graphicx} % support the \includegraphics command and options

% \usepackage[parfill]{parskip} % Activate to begin paragraphs with an empty line rather than an indent

%%% PACKAGES
\usepackage{booktabs} % for much better looking tables
\usepackage{array} % for better arrays (eg matrices) in maths
\usepackage{paralist} % very flexible & customisable lists (eg. enumerate/itemize, etc.)
\usepackage{verbatim} % adds environment for commenting out blocks of text & for better verbatim
\usepackage{subfig} % make it possible to include more than one captioned figure/table in a single float
% These packages are all incorporated in the memoir class to one degree or another...

%%% HEADERS & FOOTERS
\usepackage{fancyhdr} % This should be set AFTER setting up the page geometry
\pagestyle{fancy} % options: empty , plain , fancy
\renewcommand{\headrulewidth}{0pt} % customise the layout...
\lhead{}\chead{}\rhead{}
\lfoot{}\cfoot{\thepage}\rfoot{}

%%% SECTION TITLE APPEARANCE
\usepackage{sectsty}
\allsectionsfont{\sffamily\mdseries\upshape} % (See the fntguide.pdf for font help)
% (This matches ConTeXt defaults)

%%% ToC (table of contents) APPEARANCE
\usepackage[nottoc,notlof,notlot]{tocbibind} % Put the bibliography in the ToC
\usepackage[titles,subfigure]{tocloft} % Alter the style of the Table of Contents
\renewcommand{\cftsecfont}{\rmfamily\mdseries\upshape}
\renewcommand{\cftsecpagefont}{\rmfamily\mdseries\upshape} % No bold!

%%% END Article customizations

%%% The "real" document content comes below...

\title{chepss document}
\author{Taha Gorji M. and Parsa Darrodi}
%\date{} % Activate to display a given date or no date (if empty),
         % otherwise the current date is printed 

\begin{document}
\maketitle

\abstract{this document is about our student project to implementing a simple board game 
like chess. the rules of game explained in the similar document in persian.
the goal of project was to validating the game between two players outside.
but we do so many better things. our project give a nice view of game board and performed
actions by playesr, it let you to iterate between player actions, it can save and restore games, and may tiny features you have discover. \^ \_ \^{} }

\section{includes}
we need three heder files to include.
<stdio.h> for standard input output. <string.h> for some comparation. <sys/ioctl.h> and <unistd.h> for getting current size of terminal to a better view. 


\section{data structures}
this section contains some data types as important as they are data structures of the game. the enums implemented by \#defilne macro because they are so smaller than 4 bytes.


\section{variables}
in this section variables of the game are defined by using the data types lastly defined.


\section{constants}
in this section some constants defined for mapping data clearly. these are the symbols of pieces in the game.


\section{utility}
this section contains some simple functions for more convenience.


\section{validation}
this is your favorite section. the functions in this section will evaluate actions done by players. generally i think an action would be valid if it's legal (it means every players moves his own pieces and rules like that) and it's valid (it means the move is a valid for the type of moving piece), and it's safe (it means after the move done, player should not be checked).
but this section is not just a validation because every single error that movement could throw is distinguished by different return codes. and even this returns code are develop friendly. thanks to MRC creative protocol that saves max return code of every single validation function.

\section{game control}
this section finally contains small functions doing big jobs! these whould genelly control the game.

\section{main}
at the end main function has just to call game controllers and make it play.


\section{interaction}
every players has to move respectively. startig form white player. the turn number and the player has to move will shown before every prompt. players could enter their move in format [piece][source][dist] or just [source][dist]. they can iterate between actions by goto command and entering the turn they want to go as prompred. they can save the game and then edit that manually by using save command and entering the file path.
they can exit the game and this way they can restore any saved game by entering the file path.
they can even quit game peacfully by using quit command.
the board shown at the center of screen. at the right and left of board, dead pieces will shown. the table of performed moves will shown right of board and dead pieces according the terminal size. last move highlighted at the table and it's always vissible.
and finally error messages handled for every single possible error.




\end{document}
